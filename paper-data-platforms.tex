% Options for packages loaded elsewhere
\PassOptionsToPackage{unicode}{hyperref}
\PassOptionsToPackage{hyphens}{url}
\PassOptionsToPackage{dvipsnames,svgnames,x11names}{xcolor}
%
\documentclass[
  authoryear]{elsarticle}

\usepackage{amsmath,amssymb}
\usepackage{iftex}
\ifPDFTeX
  \usepackage[T1]{fontenc}
  \usepackage[utf8]{inputenc}
  \usepackage{textcomp} % provide euro and other symbols
\else % if luatex or xetex
  \usepackage{unicode-math}
  \defaultfontfeatures{Scale=MatchLowercase}
  \defaultfontfeatures[\rmfamily]{Ligatures=TeX,Scale=1}
\fi
\usepackage{lmodern}
\ifPDFTeX\else  
    % xetex/luatex font selection
\fi
% Use upquote if available, for straight quotes in verbatim environments
\IfFileExists{upquote.sty}{\usepackage{upquote}}{}
\IfFileExists{microtype.sty}{% use microtype if available
  \usepackage[]{microtype}
  \UseMicrotypeSet[protrusion]{basicmath} % disable protrusion for tt fonts
}{}
\makeatletter
\@ifundefined{KOMAClassName}{% if non-KOMA class
  \IfFileExists{parskip.sty}{%
    \usepackage{parskip}
  }{% else
    \setlength{\parindent}{0pt}
    \setlength{\parskip}{6pt plus 2pt minus 1pt}}
}{% if KOMA class
  \KOMAoptions{parskip=half}}
\makeatother
\usepackage{xcolor}
\setlength{\emergencystretch}{3em} % prevent overfull lines
\setcounter{secnumdepth}{5}
% Make \paragraph and \subparagraph free-standing
\ifx\paragraph\undefined\else
  \let\oldparagraph\paragraph
  \renewcommand{\paragraph}[1]{\oldparagraph{#1}\mbox{}}
\fi
\ifx\subparagraph\undefined\else
  \let\oldsubparagraph\subparagraph
  \renewcommand{\subparagraph}[1]{\oldsubparagraph{#1}\mbox{}}
\fi


\providecommand{\tightlist}{%
  \setlength{\itemsep}{0pt}\setlength{\parskip}{0pt}}\usepackage{longtable,booktabs,array}
\usepackage{calc} % for calculating minipage widths
% Correct order of tables after \paragraph or \subparagraph
\usepackage{etoolbox}
\makeatletter
\patchcmd\longtable{\par}{\if@noskipsec\mbox{}\fi\par}{}{}
\makeatother
% Allow footnotes in longtable head/foot
\IfFileExists{footnotehyper.sty}{\usepackage{footnotehyper}}{\usepackage{footnote}}
\makesavenoteenv{longtable}
\usepackage{graphicx}
\makeatletter
\def\maxwidth{\ifdim\Gin@nat@width>\linewidth\linewidth\else\Gin@nat@width\fi}
\def\maxheight{\ifdim\Gin@nat@height>\textheight\textheight\else\Gin@nat@height\fi}
\makeatother
% Scale images if necessary, so that they will not overflow the page
% margins by default, and it is still possible to overwrite the defaults
% using explicit options in \includegraphics[width, height, ...]{}
\setkeys{Gin}{width=\maxwidth,height=\maxheight,keepaspectratio}
% Set default figure placement to htbp
\makeatletter
\def\fps@figure{htbp}
\makeatother

\makeatletter
\@ifpackageloaded{caption}{}{\usepackage{caption}}
\AtBeginDocument{%
\ifdefined\contentsname
  \renewcommand*\contentsname{Table of contents}
\else
  \newcommand\contentsname{Table of contents}
\fi
\ifdefined\listfigurename
  \renewcommand*\listfigurename{List of Figures}
\else
  \newcommand\listfigurename{List of Figures}
\fi
\ifdefined\listtablename
  \renewcommand*\listtablename{List of Tables}
\else
  \newcommand\listtablename{List of Tables}
\fi
\ifdefined\figurename
  \renewcommand*\figurename{Figure}
\else
  \newcommand\figurename{Figure}
\fi
\ifdefined\tablename
  \renewcommand*\tablename{Table}
\else
  \newcommand\tablename{Table}
\fi
}
\@ifpackageloaded{float}{}{\usepackage{float}}
\floatstyle{ruled}
\@ifundefined{c@chapter}{\newfloat{codelisting}{h}{lop}}{\newfloat{codelisting}{h}{lop}[chapter]}
\floatname{codelisting}{Listing}
\newcommand*\listoflistings{\listof{codelisting}{List of Listings}}
\makeatother
\makeatletter
\makeatother
\makeatletter
\@ifpackageloaded{caption}{}{\usepackage{caption}}
\@ifpackageloaded{subcaption}{}{\usepackage{subcaption}}
\makeatother
\journal{JIMR}
\ifLuaTeX
  \usepackage{selnolig}  % disable illegal ligatures
\fi
\usepackage[]{natbib}
\bibliographystyle{elsarticle-harv}
\usepackage{bookmark}

\IfFileExists{xurl.sty}{\usepackage{xurl}}{} % add URL line breaks if available
\urlstyle{same} % disable monospaced font for URLs
\hypersetup{
  pdftitle={Conceptualizing open health data platforms for low- and middle income countries},
  pdfauthor={Daniel Kapitan; Femke Heddema; Julie Fleischer; Chris Ihure; Antragama Abbas; Steven Wanyee; XXX; John Grimes; Mark van der Graaf; Mark de Reuver},
  pdfkeywords={Analytics-on-FHIR, SQL-on-FHIR, HIE, data
platforms, LMICs, digital health},
  colorlinks=true,
  linkcolor={blue},
  filecolor={Maroon},
  citecolor={Blue},
  urlcolor={Blue},
  pdfcreator={LaTeX via pandoc}}

\setlength{\parindent}{6pt}
\begin{document}

\begin{frontmatter}
\title{Conceptualizing open health data platforms for low- and middle
income countries}
\author[1,2]{Daniel Kapitan%
%
}
 \ead{daniel@kapitan.net} 
\author[1]{Femke Heddema%
%
}

\author[1]{Julie Fleischer%
%
}

\author[3]{Chris Ihure%
%
}

\author[4]{Antragama Abbas%
%
}

\author[5]{Steven Wanyee%
%
}

\author[6]{XXX%
%
}

\author[7]{John Grimes%
%
}

\author[1]{Mark van der Graaf%
%
}

\author[4]{Mark de Reuver%
%
}


\affiliation[1]{organization={PharmAccess
Foundation},city={Amsterdam},country={the
Netherlands},countrysep={,},postcodesep={}}
\affiliation[2]{organization={Eindhoven University of
Technology},city={Eindhoven},country={the
Netherlands},countrysep={,},postcodesep={}}
\affiliation[3]{organization={PharmAccess
Kenya},city={Nairobi},country={Kenya},countrysep={,},postcodesep={}}
\affiliation[4]{organization={Delft University of
Technology},city={Delft},country={the
Netherlands},countrysep={,},postcodesep={}}
\affiliation[5]{organization={IntelliSOFT},city={Nairobi},country={Kenya},countrysep={,},postcodesep={}}
\affiliation[6]{organization={ONA},city={Nairobi},country={Kenya},countrysep={,},postcodesep={}}
\affiliation[7]{organization={Australian e-Health Research
Centre},city={Brisbane},country={Australia},countrysep={,},postcodesep={}}

\cortext[cor1]{Corresponding author}










        
\begin{abstract}
TO DO: add abstract.
\end{abstract}





\begin{keyword}
    Analytics-on-FHIR \sep SQL-on-FHIR \sep HIE \sep data
platforms \sep LMICs \sep 
    digital health
\end{keyword}
\end{frontmatter}
    
\section{Introduction}\label{sec-intro}

\subsection{Digital platforms vs.~data platforms for
healthcare}\label{digital-platforms-vs.-data-platforms-for-healthcare}

It is a widely held belief that digital technologies have an important
role to play in strengthening health systems in low- and middle income
countries (LMICs), as exemplified by the WHO global strategy on digital
health \citep{who2021global}. The adoption rate of mobile phones in
LMICs has been an important driver in implementing digital health
solutions \citep{mccool2022mobile}. Yet, there are many shortcomings and
challenges, including the current fragmentation of digital platforms and
the lack of clear-cut pathways of scaling up digital health programmes,
such that they can support sustainable and equitable change of national
health systems in LMICs
\citep{mccool2022mobile, who2019recommendations, neumark2021digital}.

A commonly used perspective to scrutinize digital health is to consider
it as a digital platform \citep{dereuver2018digital}. Digital platforms
have disrupted many sectors but have just started to make inroads into
highly regulated industries such as healthcare \citep{ozalp2022digital}.
In this light, the challenges faced by LMICs in establishing national
digital health platforms have a lot in common with those faced by high
income countries. From a technological perspective, interoperability
issues, weak integrations, siloed data repositories and overall lack of
openness are often reported as key impediments
\citep{malm-nicolais2023exploring, mehl2023fullstac}. From a societal
perspective, issues pertaining to the winner-takes-all nature of digital
platforms are hotly debated as many jurisdictions make work to ensure
these platforms indeed serve the common good of achieving universal
health coverage \citep{sharon2018when}.

Case studies on digital platforms in healthcare point to an emerging
pattern where the focus shifts from the digital platform with its
defining software and hardware components, to the data as the primary
object of interest in and of itself
\citep{ozalp2022digital, alaimo2022organizations}. This observation ties
into the proposed research agenda by de Reuver et al.~to consider data
platforms as a phenomenon distinct from digital platforms
\citep{dereuver2022openness}. Generally, data platforms inherit the
characteristics of digital platforms. For example, the economic
perspective on digital platforms stresses their multi-sidedness with
developers and consumers, while data platforms are used by data owners,
data consumers and third party solution providers. At the same time,
data platforms differ as their main offerings revolve around data. From
a market perspective, data platforms have more moderate network effects
and are more susceptible to fragmentation and heterogeneity.

Particularly relevant in the context of health data platforms (HDPs), is
the conceptualization of openness. The objective of openness, the ways
to realize openness and the risks of opening up can be considered as
elements within the ongoing debate on data spaces
\citep{otto2022designing} and data solidarity
\citep{kickbusch2021lancet, prainsack2022data, prainsack2023beyond} in
the healthcare domain, which is a current issue for high income
countries and LMICs alike. Openness is particularly relevant if we are
to realize a solidarity-based approach to health data sharing that i)
gives people a greater control over their data as active decision
makers; ii) ensures that the value of data is harnessed for public good;
and iii) moves society towards equity and justice by counteracting
dynamics of data extraction \citep{prainsack2022data}.

\subsection{From health information exchanges to health data
platforms}\label{from-health-information-exchanges-to-health-data-platforms}

This paper is motivated by the conflation of a number of developments
relevant to the design and implementation of solidarity-based HDPs in
LMICs. First, the OpenHIE framework \citep{openhie} has been adopted by
many sub-Saharan African countries \citep{mamuye2022health} as the
architectural blueprint for implementing nation-wide health information
exchanges (HIE), including Nigeria \citep{ohienigeria}, Kenya
\citep{khisif} and Tanzania \citep{tzhea2020}. These countries have, as
a matter of course, extended the framework to include ``analytics
services'' as an additional domain. The rationale for this addition is
to facilitate primary and secondary re-use of health data for academic
research, real-world evidence studies etc. which can be framed within
the context of ongoing efforts towards Findable, Accessible,
Interoperable and Reusable (FAIR) sharing of health data
\citep{guillot2023fair}. In doing so, however, we have implicitly moved
from conceptualizing digital health platforms (the original OpenHIE
specification) to health data platforms. This is problematic because the
notion of openness, which is assumed to be essential in establishing
solidarity-based approaches to data sharing, is inherently different for
a data platform compared to a digital platform.

Conceptually, the OpenHIE framework constitutes a framework for an open
digital platform. Openness for digital platforms refers to i) the use of
open boundary resources, that is, specifications for the various
healthcare specific workflows and information standards such as FHIR;
and ii) the use of open source components that are available as digital
public goods\citep{digitalpublicgoods}. If we are to use the OpenHIE
framework as an open data platform, we need to extend the standards,
technologies and architecture to include functionality for data sharing
and re-use. Distinguishing four types of data sharing
(Table~\ref{tbl-types-data-sharing}), the purpose of this paper is to
investigate how new standards and technologies that can establish
openness of health data platforms can be integrated into the OpenHIE
architecture framework. The lack of detailed specifications and
consensus of this addition to OpenHIE currently stands in the way of
development projects that aim to establish HDPs in LMICs. In addition,
we explicitly address the requirement of downward scalability as we aim
to implement health data platforms in resource constrained setting of
LMICs.

\begin{longtable}[]{@{}
  >{\centering\arraybackslash}p{(\columnwidth - 4\tabcolsep) * \real{0.1200}}
  >{\raggedright\arraybackslash}p{(\columnwidth - 4\tabcolsep) * \real{0.3200}}
  >{\raggedright\arraybackslash}p{(\columnwidth - 4\tabcolsep) * \real{0.5600}}@{}}
\caption{Types of data sharing and in relation to new standards and
technology enablers to create
openness.}\label{tbl-types-data-sharing}\tabularnewline
\toprule\noalign{}
\begin{minipage}[b]{\linewidth}\centering
\end{minipage} & \begin{minipage}[b]{\linewidth}\raggedright
\textbf{Type of data sharing}
\end{minipage} & \begin{minipage}[b]{\linewidth}\raggedright
\textbf{New standards and technology enablers to create openness}
\end{minipage} \\
\midrule\noalign{}
\endfirsthead
\toprule\noalign{}
\begin{minipage}[b]{\linewidth}\centering
\end{minipage} & \begin{minipage}[b]{\linewidth}\raggedright
\textbf{Type of data sharing}
\end{minipage} & \begin{minipage}[b]{\linewidth}\raggedright
\textbf{New standards and technology enablers to create openness}
\end{minipage} \\
\midrule\noalign{}
\endhead
\bottomrule\noalign{}
\endlastfoot
\textbf{1} & Data at the most granular (patient) level, which is
persisted and used to provide a longitudinal record. & \textbf{Bulk FHIR
API} which has by now been incorporated in all major FHIR
implementations the FHIR standard can be readily used to support type 1
data sharing to patient-level data across a patient population
\citep{mandl2020push, jones2021landscape}. \\
\textbf{2} & Aggregated data, for example statistics for policy
evaluation and benchmarking & \textbf{SQL-on-FHIR specification}
\citep{sql-on-fhir}: provides a standardised approach to make FHIR work
well with familiar and efficient SQL engines that are most commonly used
in analytical workflows. Builds on FHIRPath \citep{fhirpath} expressions
in a logical structure to specify things like column names and unnested
items. Implementations of this approach are available or forthcoming,
including open source implementations such as Pathling
\citep{grimes2022pathling} and commercial offerings like Aidbox
\citep{aidbox}. \\
\textbf{3} & Data analytics modules, that provide access to work and
access the data. & \textbf{Federated learning (FL)}
\citep{rieke2020future} and \textbf{privacy- enhancing technologies
(PETs)} \citep{scheibner2021revolutionizing, jordan2022selecting}: new
paradigms that address the problem of data governance and privacy by
training algorithms collaboratively without exchanging the data itself.
Models can be trained on combined datasets and made available as open
source artifacts for decision support. Data analysts can use FL and PETs
to work with the data in a collaborative, decentralized fashion. \\
\textbf{4} & Trained models that have been derived from the data and can
be used stand-alone for decision support. & \textbf{ONNX} \citep{onnx}:
an open format built to represent machine learning models. \\
\end{longtable}

\section{Methods}\label{methods}

In the paper, we present a design that extends the OpenHIE specification
to include the four types of data sharing mentioned above. Using the
full-STAC approach \citep{mehl2023fullstac} we combine open standards,
open technologies and open architectures into a coherent modular HDP
platform that can be configured and re-used across a variety of
use-cases. Subsequently, we employ a formative, naturalistic evaluation
to assess the technical risk and efficacy of the design
\citep{venable2016feds}. Given that it is prohibitively expensive to
evaluate with real users and real systems in the real setting, we aim to
minimize technological risks and maximize the efficacy of the design by
considering three real-world examples of health data platoform in LMICs,
namely i) the OpenHIM platform
(\url{https://jembi.gitbook.io/openhim-platform/}); ii) the ONA Canopy
platform \url{https://ona.io/home/services/canopy/}; and iii) the
MomCare platform (\url{https://health-data-commons.pharmaccess.org}. As
part of our design research, we have taken a narrative approach in
surveying existing scientific studies on health data platforms, focusing
on the seminal reports and subsequently searching forward citations. In
addition, we have searched the open source repositories (most notably
GitHub) and the online communities (OpenHIE community, FHIR community)
to search for relevant open standards, technologies and architectures.
This paper should not be considered as a proper systematic review.

\section{Design}\label{design}

\subsection{Open standards: using FHIR as the common data
model}\label{open-standards-using-fhir-as-the-common-data-model}

The recent convergence to FHIR as the de facto standard for information
exchange has fuelled the development of OpenHIE. FHIR is currently used
both for routine healthcare settings\citep{ayaz2021fast} and clinical
research settings \citep{duda2022hl7, vorisek2022fast} and is
increasingly being used in LMICs as well. The guidelines and standards
of the African Union explicitly state FHIR is to be used as the
messaging standard \citep{2023african}. The FHIR-native OpenSRP platform
\citep{mehl2020open} has been deployed in 14 countries targeting various
patient populations, amongst which a reference implementation of the WHO
antenatal and neonatal care guidelines for midwives in Lombok, Indonesia
\citep{summitinstitutefordevelopment2023bunda, kurniawan2019midwife}. In
India, FHIR is used as the underlying technology for the open Health
Claims Exchange protocol specification, which has been adopted by the
Indian government as the standard for e-claims handling \citep{hcx}.
This range of utilizations showcase the standards' widespread
applicability. The proceedings of the OpenHIE conference 2023 attest to
the fact that FHIR and open source technologies are embraced as critical
enablers in implementing health information exchanges in LMICs
\citep{ohie23}.

However, despite the increased use of FHIR as a common data model,
various studies have investigated its merits and performance vis-a-vis
other healthcare standards. Comparisons between OpenEHR, ISO 13606, OMOP
and FHIR have been made
\citep{ayaz2023transforming, mullie2023coda, rinaldi2021openehr, cremonesi2023need, sinaci2023data}.
A study involving 10 experts comparing OpenEHR, ISO 13606 and FHIR
concluded that i) these three standards are functionally and technically
compatible, and therefore can be used side by side; and that ii) each of
these standards have their strengths and limitations that correlate with
their intended use as summarized in the Table~\ref{tbl-comparison}.

\begin{table}

\caption{\label{tbl-comparison}Comparison of OpenEHR, ISO 13606 and FHIR
standards}

\centering{

\includegraphics{images/comparison-ehr-standards.png}

}

\end{table}%

For an infectious diseases dataset with a limited scope, OpenEHR, OMOP
and FHIR have been compared and found all to be equally suitable
\citep{rinaldi2021openehr}. Comparing OMOP and FHIR, the latter has been
found to support more granular mappings required for analytics and was
therefore chosen as the standard for the CODA project
\citep{mullie2023coda}.

Although FHIR was originally designed only for exchange between systems,
we propose to use it as the common data model for the design presented
here for the following reasons:

\begin{itemize}
\tightlist
\item
  Industry adoption has significantly increased, as exemplified by
  FHIR-based offering by major cloud providers such as Google, Azure and
  AWS. Also, Africa CDC has explicitly chosen FHIR as the preferred
  standard;
\item
  The widespread availibility of the Bulk FHIR API
  \citep{mandl2020push, jones2021landscape} enables bulk, fileb-based
  nbatchwise processing for analytics using the lakehouse architecture
  as detailed in the next section;
\item
  The concept of FHIR Profiles allow localisation to tailor the standard
  to a specific use case. A profile defines rules, extensions, and
  constraints for a resource. We posit that the possible penalty of this
  flexibility, namely having to manage different FHIR versions and/or
  profiles, is less of an issue in the context of LMICs where first
  priority is to exchange datasets such as the International Patient
  Summary (IPS) that are less complex compared to the requirements for
  high income countries;
\item
  Being based on webstandards, the FHIR standard lends itself best for
  further separation of concerns as envisioned by the composable data
  stack. This is an important enabler for the downward scalability of
  the solution;
\item
  With its inherent, graph-like nature, FHIR can be readily incorporated
  into the principles of FAIR data sharing, where FHIR-based data
  repositories can be integrated in an overarching netwerk of FAIR data
  stations \citep{sinaci2023data, pedrera-jimenez2023can}.
\end{itemize}

Possible risks pertaining to the use of FHIR as the common data model,
most notably the possible incompatibilities and/or high costs of
maintenance in supporting different versions, will be addressed in the
Discussion.

\subsection{Open architecture: extending OpenHIE framework with a
lakehouse data
platform}\label{open-architecture-extending-openhie-framework-with-a-lakehouse-data-platform}

Data management and analytics platforms have undergone significant
changes since the first generation of data warehouses were introduced.
Recent studies have shown that the current practice has converged
towards the lakehouse as one of the most commonly used solution designs
\citep{armbrust2021lakehouse, hai2023data, harby2022data}. Lakehouses
typically have a zonal architecture \citep{hai2023data} where data is
ingested from the source systems in bulk (E), delivered to storage with
aligned schemas (L) and transformed into a format ready for analysis
(T). The discerning characteristic of the lakehouse architecture is its
foundation on low-cost and directly-accessible storage that also
provides traditional analytical DBMS management and performance features
such as ACID transactions, data versioning, auditing, indexing, caching,
and query optimization \citep{armbrust2021lakehouse}. Lakehouses thus
combine the key benefits of data lakes and data warehouses: low-cost
storage in an open format accessible by a variety of systems from the
former, and powerful management and optimization features from the
latter.

With respect to current implementations of lakehouse data platforms, we
observe a proliferation of tools with as yet limited standards to
improve technical interoperability. In the analysis of Pedreira et al.
\citep{pedreira2023composable} the requirement for specialization in
data management systems has evolved faster than our software development
practices. This situation has created a siloed landscape composed of
hundreds of products developed and maintained as monoliths, with limited
reuse between systems. It has also affected the end users, who are often
required to learn the idiosyncrasies of dozens of incompatible SQL and
non-SQL API dialects, and settle for systems with incomplete
functionality and inconsistent semantics. To remedy this, Pedreira et
al.~call to (re-)design and implement modern data platforms in terms of
a `composable data stack' as a means to decrease development and
maintenance cost and pick-up the speed of innovation.

While the lakehouse architecture separates the concerns of compute and
storage, the composable data stack takes the separation of concerns is
taken one step further. A composable data system
(Figure~\ref{fig-composable-data-stack}), not only separates the storage
(layer 3) and execution (layer 2), but also separates the user interface
(layer 1) from the execution engine by introducing standards for
Intermediate Representation (standard A) and Connectivity (standard B).
The composable data stack can be implemented with current open source
technologies (Figure~\ref{fig-cds-examples}). As an example, the Ibis
user interface is currently sufficiently mature to offer a standardized
dataframe interface to 19 different execution engines.

\begin{figure}

\centering{

\includegraphics{images/composable-data-stack.png}

}

\caption{\label{fig-composable-data-stack}Composable data stack}

\end{figure}%

\begin{figure}

\centering{

\includegraphics{images/composable-data-stack-implementation.png}

}

\caption{\label{fig-cds-examples}Examples of implementations of the
composable data stack}

\end{figure}%

The premise of separating the user interface from the execution engine
is directly related to the key objective of the SQL-on-FHIR project
(\url{https://build.fhir.org/ig/FHIR/sql-on-fhir-v2/}), namely to make
large-scale analysis of FHIR data accessible to a larger audience,
portable between systems and to make FHIR data work well with the best
available analytic tools, regardless of the technology stack. However,
as FHIR is represented as a graph of resources, with detailed semantics
defined for references between resources, data types, terminology,
extensions, and many other aspects of the specification; to use FHIR
effectively analysts require a thorough understanding of the
specification. Most analytic and machine learning use cases require the
preparation of FHIR data using transformations and tabular projections
from its original form. The task of authoring these transformations and
projections is not trivial and there is currently no standard mechanisms
to support reuse.

The solution of the SQL-on-FHIR project is to provide a specification
for defining tabular, use case-specific views of FHIR data. The view
definition and the execution of the view are separated, in such a way
that the definition is portable across systems while the execution
engine (called runners) are system-specific tools or libraries that
apply view definitions to the underlying data layer, optionally making
use of annotations to optimize performance.

Our proposal for extending HIE is as follows.

\begin{itemize}
\tightlist
\item
  add diagram, explicitly adding the component
\item
  fit in structure of OpenHIE specification
\item
  check which workflows are related to analytics
\end{itemize}

Following Hai we take a subset of the core functionalities of a data
lake. Because the SHR is the key source, don't need full stack of
ingestion of various sources.

\includegraphics{./images/hie.png}

\begin{itemize}
\tightlist
\item
  Query-driven data discovery
\end{itemize}

\subsection{Open technologies: available digital public
goods}\label{open-technologies-available-digital-public-goods}

Many components of the OpenHIE specification are now available as a
digital public goods. Table~\ref{tbl-digital-public-goods} lists
components that are currently available for implementing the OpenHIE
framework using open source, digital public goods that are compliant
with the FHIR standard, illustrating the maturity of this ecosystem and
development community. With the launch of the Instant OpenHIE
configuration toolkit\citep{InstantOpenHIEv2}, it has become easier to
set up, explore and develop HIEs thereby reducing costs and skills
required for software developers to deploy an OpenHIE architecture for
quicker solution testing and as a starting point for faster production
implementation and customisation. Several frameworks are available that
offer a set of preconfigured components out of the box, such as for
example:

\begin{itemize}
\tightlist
\item
  the ``Open Smart Register Platform'' (OpenSRP), that focuses on
  providing a mobile-first platform, including a FHIR native app
  designed to support the WHO Smart Guidelines
\item
  the OpenHIM Platform, a reference implementation of the Instant
  OpenHIE framework, providing an easy way to set up, manage and operate
  various HIE configurations
\end{itemize}

\begin{table}

\caption{\label{tbl-digital-public-goods}Overview of current open source
implementations of components included in the OpenHIE specification that
are FHIR-compatible. The category Analytics Services is not a part of
the original OpenHIE and is discussed in the paper. Point-of-Service
systems are excluded for brevity. A systematic review of such digital
public goods is beyond the scope of this document.}

\centering{

\includegraphics{images/digital-public-goods.png}

}

\end{table}%

\section{Evaluation}\label{evaluation}

\subsection{OpenHIM platform}\label{openhim-platform}

\begin{itemize}
\tightlist
\item
  View generation in different packages, not portable

  \begin{itemize}
  \tightlist
  \item
    Logstash for bulk
  \item
    Kafka for streaming
  \end{itemize}
\item
\end{itemize}

\subsection{ONA Canopy}\label{ona-canopy}

\subsection{Momcare programme}\label{momcare-programme}

MomCare was launched in Kenya
\citep{huisman2022digital, sanctis2022maintaining} and Tanzania
\citep{shija2021access, mrema2021application} in 2017 and 2019
respectively, with the objective to create transparency on the journeys
of pregnant mothers. The programme is built on three pillars: journey
tracking, quality support and a mobile wallet
(Figure~\ref{fig-momcare}). MomCare distinguishes two user groups:
mothers are supported during their pregnancy through reminders and
surveys, using SMS as the digital mode of engagement. Health workers are
equipped with an Android-based application, in which visits, care
activities and clinical observations are recorded. Reimbursements of the
maternal clinic are based on the data captured with SMS and the app,
thereby creating a conditional payment scheme, where providers are
partially reimbursed up-front for a fixed bundle of activities,
supplemented by bonus payments based on a predefined set of care
activities.

\begin{figure}

\centering{

\includegraphics{images/momcare.png}

}

\caption{\label{fig-momcare}Overview of the Momcare programme}

\end{figure}%

\section{Discussion}\label{discussion}

\subsection{Openness of data
platforms}\label{openness-of-data-platforms}

We specifically address the notion of openness of HDPs in LMICs in terms
of the design-related questions put forward by de Reuver at al.11:

\begin{itemize}
\tightlist
\item
  Object of openness: what data-related resources should data platforms
  make available when opening up (e.g.~data, data products, datadriven
  insights, analytics modules)? Which user groups derive value from
  accessing data-related resources from data platforms (e.g.~data
  providers, data users, intermediaries, developers)?
\item
  Unit of analysis: what is platform-to-platform openness in the context
  of data platforms, given the expectation that different HDPs will
  emerge at various aggregation levels? How do we distinguish
  meta-platforms, forking, and platform interoperability?
\item
  Risk of openness: What are the novel (negative) implications of
  opening up data platforms? How can reflexivity in design help
  providers to resolve the negative implications of openness?
\end{itemize}

\subsection{Comparison with HMIS
component}\label{comparison-with-hmis-component}

\begin{itemize}
\tightlist
\item
  Workflow requirements: Report aggregate data (link): receiver is HMIS,
  mADX
\item
  Functional requirements:
  https://guides.ohie.org/arch-spec/openhie-component-specifications-1/openhie-health-management-information-system-hmis
\end{itemize}

Requirements are similar, but implementation differs: Datamodel is
non-FHIR, focused on DataValue, which conceptually equates to FHIR
Measure

\subsection{FHIR and FAIR}\label{fhir-and-fair}

\begin{itemize}
\tightlist
\item
  How does FHIR relate to approaches taken by the FAIR community, which
  tend to take more an approach of using knowledge graphs. For example,
  VODAN Africa \citep{gebreslassie2023fhir4fair, purnamajati2022data}.
\item
  FAIR principles vs FHIR graph: is FHIR a FAIR Data Object
\end{itemize}

\subsection{Attribute-based access
control}\label{attribute-based-access-control}

\begin{itemize}
\tightlist
\item
  TO DO: if you have generated flattened SQL tables, how are you going
  to manage security?
\item
  Cerbos, attribute based on lineage or anonymized tables
\end{itemize}

\section{Abbreviations}\label{abbreviations}

\begin{longtable}[]{@{}
  >{\raggedright\arraybackslash}p{(\columnwidth - 2\tabcolsep) * \real{0.5000}}
  >{\raggedright\arraybackslash}p{(\columnwidth - 2\tabcolsep) * \real{0.5000}}@{}}
\toprule\noalign{}
\endhead
\bottomrule\noalign{}
\endlastfoot
ELT & Extract, Load and Transform \\
FAIR & Findable, Accessible, Interoperable and Reusable \\
FHIR & Fast Healthcare Interoperability Resources \\
HDP & Health data platform, explicitly differentiated from health
digital platform \\
HIE & Health Information Exchange \\
FL & Federated learning \\
LMIC & Low- and middle income countries \\
PET & Privacy-enhancing technologies \\
\end{longtable}


  \bibliography{pharmaccess.bib}


\end{document}
